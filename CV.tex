
\documentclass[10pt,letterpaper,sans]{moderncv}

% moderncv themes
%\moderncvtheme[blue]{casual}                 % optional argument are 'blue' (default), 'orange', 'red', 'green', 'grey' and 'roman' (for roman fonts, instead of sans serif fonts)
\moderncvtheme[blue]{classic}                    % idem

% character encoding
\usepackage[utf8]{inputenc}                        % replace by the encoding you are using

% adjust the page margins
\usepackage[scale=0.92]{geometry}
\recomputelengths                                       % required when changes are made to page layout lengths

\usepackage{microtype}
\usepackage[method=mhchem]{chemmacros}
\usepackage{multicol}
\usepackage{graphicx}

% personal data
\firstname{Brian}
\familyname{Beatty}
\title{Curriculum Vit\ae}               % optional, remove the line if not wanted
\address{\hspace{-1in}19 Twin Circle Drive}{Westport, CT 06880, USA}    % optional, remove the line if not wanted
\mobile{Mobile:~+34~677~684~381}
\phone{Phone:~+1~(203)~221~0567}
%\email{Beatty.Brian@gmail.com}
\email{Brian.R.Beatty@Drexel.edu}                      % optional, remove the line if not wanted
%\extrainfo{additional information (optional)}    % optional, remove the line if not wanted
                        % '64pt' is the height the picture must be resized to and 'picture' is the name of the picture file; optional, remove the line if not wanted


\nopagenumbers{}                             % uncomment to suppress automatic page numbering for CVs longer than one page


%----------------------------------------------------------------------------------
%            content
%----------------------------------------------------------------------------------
\begin{document}

\maketitle

%\vspace{-1.0cm}

\section{Education}

\cventry{2011-Present}{M.S. Studies in Materials Science (Dual Degree)}{Politecnico di Milano}{Milan}{Italy}{Selected as one of Drexel University's first participants in the newly formed EAGLES International Exchange Program sponsered by the EU Commision and the US Department of Education and Cultural Affairs}  

\cventry{2007-Present}{B.S./M.S. Studies in Materials Science (Major)}{Drexel University}{Philadelphia}{PA}{Cualified to Graduate with Honors\\Cumulative GPA: 3.56\newline{} Junior/Senior GPA:  3.92\\Selected as one of 15 STAR (Students Tackling Advanced Research) Scholars\\Awarded A.J. Drexel Dean's Scholarship and William F. Mitchell SuperNOVA Scholarship}

%%%%%%%%%%%% MASTER'S THESIS %%%%%%%%%%%%%

\section{Master's thesis}
\cvline{Title}{\emph{Atomic Layer Deposition of Perovskite Oxide Thin Films}}
\cvline{Supervisors}{Dr. Jonathan E. Spanier --- \textit{Drexel University, Philadelphia, PA, USA}}
\cvline{}{Dr. Carlo S. Casari --- \textit{Politecnico di Milano, Milano, Italia}}
%\cvline{}{Dr. XYZ --- \textit{Universidad Polit\'ecnica de Madrid, Madrid, Espa\~na}}
%
\cvline{Description}{\small Atomic Layer Deposition (ALD) has been shown to be capable of depositing ultra-thin films (<100 nm) of perovskite oxides. These are technologically valuable due to their wide range of properties, ranging from ferroelectricity and ferromagnetism to superconductivity. With the use of ALD, it becomes possible to deposit thin and conformal coatings of these materials to three-dimensional surfaces with the ability to carefully control thickness with sub-nanometer resolution. Research work focuses on the lead titanate (PbTiO$_3$) end group of the lead zirconate titanate (PbTi$_{\mathrm{x}}$Zr$_{1-\mathrm{x}}$O$_{3}$) system.}

%%%%%%%%%% EXPERIENCE %%%%%%%%%%%%%%%%%

\section{Experience}

\cventry{Summer 2012}{Thesis Research}{Universidad Polit\'ecnica de Madrid}{Madrid}{Spain}{Completed remaining experiments and data analysis for M.S. thesis. Authored final M.S. thesis document.\\Defended the project in collaboration with research groups from Spain, Italy, and the USA.}

%\cventry{2009--2011}{Cooperative Work Experience}{MesoMaterials Laboratory}{Drexel University}{}{Three six month periods (Spring and Summer) of full-time work as a student researcher.\newline{}Focused research on application of atomic layer deposition to ferroelectric oxide films. 
%\begin{itemize}%
%\item 2011
%  \begin{itemize}%
%  \item Sub-achievement (a);
%  \item Sub-achievement (b);
%  \item Sub-achievement (c);
%  \end{itemize}
%\item 2010
%  \begin{itemize}%
%  \item Sub-achievement (a);
%  \item Sub-achievement (b);
%  \item Sub-achievement (c);
%  \end{itemize}
%\item 2009
%  \begin{itemize}%
%  \item Sub-achievement (a);
%  \item Sub-achievement (b);
%  \item Sub-achievement (c);
%  \end{itemize}
%\end{itemize}
%}

\cventry{2009--2011}{Research Experience}{MesoMaterials Laboratory (MML)}{Drexel University}{}{Focused research on application of Atomic Layer Deposition to ferroelectric oxide thin films.%
	\begin{itemize}
	\item\emph{Co-Op Experiences} were 6 month periods (Spring/Summer) of full time employment at MML.%
	\item\emph{Student Research} was performed during periods when simultaneously attending courses.
%	\end{itemize} 
%\begin{multicols}{2}
%\textbf{Co-Op Experiences}
%\begin{itemize}%
%\item 2011
%  \begin{itemize}%
%     \item Sub-achievement (a)
%     \item Sub-achievement (b)
%     \item Sub-achievement (c)
%  \end{itemize}
%\item 2010
%  \begin{itemize}%
%     \item Development of ellipsometry analysis procedure
%     \item Designed new sub-system for ALD reactor
%     \item Lead international group in ALD oxide deposition
%  \end{itemize}
%\item 2009
%  \begin{itemize}%
%     \item Assumed Leadership of PbTiO$_{3}$ Project
%     \item Assigned to Management of ALD Reactor
%    % \item Sub-achievement (c)
%  \end{itemize}
%\end{itemize}
%
%\columnbreak
%
%\textbf{Student Research}
%\begin{itemize}%
%\item 2011
%  \begin{itemize}%
%     \item Sub-achievement (a)
%     \item Sub-achievement (b)
%     \item Sub-achievement (c)
%  \end{itemize}
%\item 2010
%  \begin{itemize}%
%     \item 
%     \item Sub-achievement (c)
%  \end{itemize}
%\item 2009
%  \begin{itemize}%
%     \item Statistics-based design of experiments
%     \item Model- and simulation-driven process optimization
%  \item Determination of alternative reaction materials
%  \end{itemize}
%\end{itemize}
%\end{multicols}\vspace{0.25em}
%}
\begin{itemize}%
\item 2011
  \begin{itemize}%
     \item Collaborated in development of ellipsometry analysis procedure for thin films
     \item Designed new sub-system for ALD reactor, improving capabilities and minimizing precursor consumption
     \item Lead international group in ALD oxide deposition, focused on both \ce{PbTiO3} and \ce{BiFeO3} systems
  \end{itemize}
\item 2010
  \begin{itemize}%
     \item Applied statistics-based design of experiments to solve a problem with interfering variables
     \item Model- and simulation-driven process optimization to select precursors for improving yields
     \item Determination of alternative reaction materials
  \end{itemize}
\item 2009
  \begin{itemize}%
     \item Assumed leadership of \ce{PbTiO3} project
     \item Assigned to management and upkeep of ALD reactor
    % \item Sub-achievement (c)
  \end{itemize}
\end{itemize}
\end{itemize}
}
\cventry{Fall, 2011}{Abstract Accepted to MRS Fall Session}{}{}{}{On results of oriented lead titanate thin film deposition via ALD onto single crystalline surfaces.}

\cventry{Summer 2008}{STAR: Summer Research Experience}{MesoMaterials Laboratory}{Drexel University}{}{%
%Ten week period spent learning to organize and develop a personal research project. 
Conceptualized and organized a personal reserah project that developed into a long term research and M.S. Thesis funded by the ARO.  
 \begin{itemize}%
  \item Data analysis (using Matlab and Igor software packages)
  \item Training and over 150 hours of experience on various characterization tools
  \end{itemize}
}    

\cventry{2007--2008}{Training in Atomic Layer Deposition}{MesoMaterials Laboratory}{Drexel University}{}{Worked under graduate mentor Rahul Joseph researching application of advanced thin film deposition methods. Trained on operation and maintenance of ALD reactor, } 


%%%%
%
%\rule{\textwidth}{1pt}
%
%\cventry{Summer 2012}{Thesis Research}{Universidad Polit\'ecnica de Madrid}{Madrid}{Spain}{Three month period to be spent completing remaining experiments and authoring final M.S. thesis document and defense presentation.}
%
%\cventry{Fall, 2011}{Abstract Accepted to MRS}{}{}{}{On results of oriented lead titanate thin film deposition via ALD onto single crystalline surfaces.} 
%            
%\cventry{Spring, Summer 2011}{Cooperative Work Experience}{MesoMaterials Laboratory}{Drexel University}{}{Three six month periods (Spring, Summer) of full-time work as a student researcher. Focused research on application of atomic layer deposition to ferroelectric oxide films. }               
%
%\cventry{2011}{Student Research}{MesoMaterials Laboratory}{Drexel University}{}{XXXX designing and conducting experiments concurrently while attending classes. Primary goals: Utilizing statistics to improve experimental designs; Optimizing process conditions based on predictive modeling and thermodynamic simulation; Identification of alternative chemical precursors to alleviate operational limitations. }
%
%\cventry{Spring, Summer 2010}{Cooperative Work Experience}{MesoMaterials Laboratory}{Drexel University}{}{Three six month periods (Spring, Summer) of full-time work as a student researcher. Focused research on application of atomic layer deposition to ferroelectric oxide films. }   
%
%\cventry{2010}{Student Research}{MesoMaterials Laboratory}{Drexel University}{}{XXXX designing and conducting experiments concurrently while attending classes. Primary goals: Utilizing statistics to improve experimental designs; Optimizing process conditions based on predictive modeling and thermodynamic simulation; Identification of alternative chemical precursors to alleviate operational limitations. }
%
%\cventry{Spring, Summer 2009}{Cooperative Work Experience}{MesoMaterials Laboratory}{Drexel University}{}{Three six month periods (Spring, Summer) of full-time work as a student researcher. Focused research on application of atomic layer deposition to ferroelectric oxide films. }   
%
%\cventry{2008-2009}{Student Research}{MesoMaterials Laboratory}{Drexel University}{}{XXXX designing and conducting experiments concurrently while attending classes. Primary goals: Utilizing statistics to improve experimental designs; Optimizing process conditions based on predictive modeling and thermodynamic simulation; Identification of alternative chemical precursors to alleviate operational limitations. }
%
%\cventry{Summer 2008}{STAR: Summer Research Experience}{MesoMaterials Laboratory}{Drexel University}{}{Ten week period spent learning to organize and develop a personal research project. Focused on statistics-based design of experiments; data analysis using Matlab and Igor software packages and custom scripts; received training for and logged over 200 hours on characterization equipment, including: ellipsometry, scanning electron microscopy and EDS spectroscopy, X-Ray diffractometry, and Rutherford Backscattering Spectroscopy.}    
%
%\cventry{2007--2008}{Training in Atomic Layer Deposition}{MesoMaterials Laboratory}{Drexel University}{}{Worked under graduate mentor Rahul Joseph researching application of advanced thin film deposition methods. Trained on operation and maintenance of ALD reactor, }


%%%%%%%% PUBLICATIONS %%%%%%%%%%%%%%%
              
\section{Publications}

\cventry{April, 2012}{Shape-Controlled Vapor-Transport Growth of Tellurium Nanowires}{\normalfont Christopher J. Hawley, Brian R. Beatty, Guannan Chen, and Jonathan E. Spanier}{Crystal Growth \& Design \textbf{2012} \emph{12}~(6), 2789--2793}{}{}

\cventry{March, 2012}{The Ferroelectric Field Effect within an Integrated Core/Shell Nanowire}{\normalfont Stephen S. Nonnenmann, Mohammad A. Islam, Brian R. Beatty, Eric M. Gallo, Terrence McGuckin, and Jonathan E. Spanier}{Adv. Funct. Mater. \textbf{2012}}{}{}

%%%%%%%% EQUIPMENT TRAINING %%%%%%%%%%%%%%%%

\section{Equipment Experience}
\cventry{\textbf{Thin Film Deposition}}{\textmd{Atomic Layer Deposition (ALD)}}{\textup{Chemical Vapor Deposition (CVD), Pulsed Laser Deposition (PLD), \mbox{Molecular} Beam Epitaxy (MBE), Sol-Gel Deposition, Thermal Evaporation, E-Beam Evaporation, R.F. Sputtering}}{}{}{}\vspace{1.5em}

\cventry{\textbf{Film Analysis Methods}}{\textmd{X-Ray Diffractometry (XRD)}}{\textup{X-Ray Reflectivity (XRR), Scanning Electron Microscopy (SEM), Energy-Dispersive X-Ray Spectroscopy (EDXS), X-Ray Fluorescence Spectroscopy (XRF), Ellipsometry, Rutherford Backscattering \mbox{Spectroscopy}, Photoluminescence}}{}{}{}\vspace{1.5em}

\cventry{\textbf{Chemical Analysis}}{\textmd{Fourier-Transform Infrared Spectroscopy (FTIR)}}{\textup{Differential Scanning Calorimetry (DSC), \mbox{Thermo-gravimetric} Analysis (TGA), Gas-Chromatography/Mass-Spectroscopy (GC-MS)}}{}{}{}
\vspace{1.5em}

\cventry{\textbf{Electrical Characterization}}{\textmd{Hall Effect measurements}}{\textup{Cyclicvoltammetry (CV), Piezoelectric/Ferroelectric measurements, Piezoelectric force microscopy (PFM)}}{}{}{}

%%%%%%%%%%%%%%%%%%%%%%%%%%%%%%%%%%%%%

%\section{Equipment Training} 
%\cvcomputer{\textbf{Thin Film Deposition}}{Atomic Layer Deposition (ALD), Chemical Vapor Deposition (CVD), Pulsed Laser Deposition (PLD), Molecular Beam %Epitaxy (MBE)} {}{}
%\cvcomputer{\textbf{Film Quantification}}{, X-Ray Reflectivity (XRR), Scanning Electron Microscopy (SEM), Energy-Dispersive X-Ray Spectroscopy (EDXS), X-Ray %Fluorescence (XRF), Ellipsometry, Rutherford Backscattering Spectroscopy (RBS)}{}{} 
%\cvcomputer{\textbf{Chemical and Thermal Analysis}}{Fourier-Transform Infrared Spectroscopy (FTIR), Differential Scanning Calorimetry (DSC),   Thermo-%gravimetric Analysis (TGA)}{}{} 
%\cvcomputer{}{} {}{}
%\vspace{10 mm}

\section{Computer skills} 
\cvcomputer{Languages}{MATLAB, Maple, \LaTeX, Igor} {}{}
\cvcomputer{Tools}{Origin, Igor, LabView, FilmWizard,\\Microsoft Office Suite, Adobe Suite} {} {}
\cvcomputer{Platforms}{Mac OS, Windows}  {}{} 
%\cvcomputer{}{} {}{}
%\vspace{10 mm} 

\section{Spoken Languages}
\cvlanguage{English}{Native Speaker}{}  \vspace{0.5em}

\cvlanguage{Spanish}{\normalfont%
				{\bfseries Speaking}: Moderate \\%
				{\bfseries Reading}: Moderate\\%
				{\bfseries Writing}: Moderate}{} \vspace{0.5em}

\cvlanguage{Italian}{\normalfont%
				{\bfseries Speaking}: Elementary \\%
				{\bfseries Reading}: Elementary\\%
				{\bfseries Writing}: Elementary}{}



%%%%%%%%%%%%%%%%%%%%
%%%%%%%%%%%%%%%%%%%%
%%%%%%%%%%%%%%%%%%%%

\section{Personal Interests}
\cvitem{Electronics}{\small Design and synthesis of novel materials for IC applications, particularly those leveraging unique nanoscale properties} 
\cvitem{Energy}{\small Nanoscale structures for high-efficiency photovoltaic devices. }
\cvitem{Culinary Arts}{\small Personal hobby.}

\closesection{}                   % needed to renewcommands
\renewcommand{\listitemsymbol}{-} % change the symbol for lists


%\section{Extra 1}
%\cvlistitem{Item 1}
%\cvlistitem{Item 2}
%\cvlistitem[+]{Item 3}            % optional other symbol

%\section{Extra 2}
%\cvlistdoubleitem[\Neutral]{Item 1}{Item 4}
%\cvlistdoubleitem[\Neutral]{Item 2}{Item 5}
%cvlistdoubleitem[\Neutral]{Item 3}{}

% Publications from a BibTeX file
%\nocite{*}
%\bibliographystyle{plain}
%\bibliography{publications}       % 'publications' is the name of a BibTeX file



%-----       letter       ---------------------------------------------------------
\newpage
% recipient data
\recipient{Human Resources Dept.}{%
	TriQuint Semiconductor, Inc.\\%
	2300 NE Brookwood Parkway\\%
	Hillsboro, Oregon 97124}
\date{\today}
\opening{Dear Sir or Madam,}
\closing{\vspace{1em}Sincerely,\vspace{-0.5cm}}
\enclosure{Curriculum Vit\ae{}}
\makelettertitle

I am interested in the position of Materials Engineer in your Research \& Development department, particularly the position focusing on Thin Film depositions via MOCVD and ALD. Included with this letter is a copy of my Curriculum Vit\ae{}, and I believe that you will find me to be a prime candidate for this position. 

I have significant experience in the fields relevant to this position, especially in the development and application of Atomic Layer Deposition (ALD) processes. 

\vspace{1.5em}
Your prime requirements for the applicant mentioned on the posting included the following:

\textbf{Position Requirements:}
\begin{itemize}
	\item Experience with ALD for various types of thin films
	\item Knowledge of epitaxial deposition methods for semiconductors.
	\item Training and experience in materials characterization techniques
\end{itemize}
\vspace{1.5em}

With a look at my CV you can see that I have matched or exceeded these qualifications.

\textbf{Personal Qualifications:}
\begin{itemize}
	\item Over four years of personal and self-driven experience on developing novel ALD deposition methods; culminating in my M.S. Thesis
	\item Close collaboration and partnership with colleagues working in CVD of advanced materials (semiconductors, nanowires, etc.)
	\item Extensive and varied training in an array of advanced characterization methods
\end{itemize}
\vspace{1.5em}

Through the process of obtaining my degree I have also had to work in close collaboration with groups of varied backgrounds, was an integral part of numerous international partnerships, and have assisted on various projects with industry partners.

In short, I feel that I am a prime candidate for this position, and appreciate you taking the time to review my credentials and experience. Thank you very much in advance for your consideration of my application.

\makeletterclosing


\end{document}
%% end of file `template_en.tex'.
