
\documentclass[11pt, letterpaper]{moderncv}
\documentclass[11pt, a4paper]{moderncv}

% moderncv themes
%\moderncvtheme[blue]{casual}                 % optional argument are 'blue' (default), 'orange', 'red', 'green', 'grey' and 'roman' (for roman fonts, instead of sans serif fonts)
\moderncvtheme[blue]{classic}                    % idem

% character encoding
\usepackage[utf8]{inputenc}                        % replace by the encoding you are using

% adjust the page margins
\usepackage[scale=.96]{geometry}
\usepackage[scale=.90]{geometry}
\recomputelengths                                       % required when changes are made to page layout lengths

%\usepackage{microtype}
% personal data
\firstname{Brian}
\familyname{Beatty}
\title{Curriculum Vit\ae}               % optional, remove the line if not wanted
\address{\hspace{-1in}Via Luciano Zuccoli, 19}{Milano - MI - Italy}    % optional, remove the line if not wanted
\mobile{Mobile +39 (388) 161-2448}
\address{\hspace{-1in}Via Luciano Zuccoli, 19}{Milano, MI, Italia}    % optional, remove the line if not wanted
\mobile{Mobile~+39~(388)~161-2448}
%\email{Beatty.Brian@gmail.com}
\email{Brian.R.Beatty@drexel.edu}                      % optional, remove the line if not wanted
%\extrainfo{additional information (optional)}    % optional, remove the line if not wanted
                        % '64pt' is the height the picture must be resized to and 'picture' is the name of the picture file; optional, remove the line if not wanted

\nopagenumbers{}                             % uncomment to suppress automatic page numbering for CVs longer than one page


%----------------------------------------------------------------------------------
%            content
%----------------------------------------------------------------------------------
\begin{document}
\maketitle

\vspace{-0.5cm}
\section{Education}

\cventry{2011-Present}{M.S. Studies in Materials Science (Dual Degree)}{Politecnico di Milano}{Milan}{Italy}{Attended for academic year 2011-2012 as part of EAGLES International Exchange Program}  

\cventry{2007-Present}{B.S./M.S. Studies in Materials Science (Major)}{Drexel University}{Philadelphia}{PA}{Cumulative GPA: 3.56\\ Previous Six Quarters GPA:  3.92}  


%%%%%%%%%%%% MASTER'S THESIS %%%%%%%%%%%%%

\section{Master's thesis}
\cvline{Title}{\emph{Atomic Layer Deposition of Perovskite Oxide Thin Films}}
\cvline{Supervisors}{Dr. Jonathan Spanier --- \textit{Drexel University, Philadelphia, PA, USA}}
\cvline{}{Dr. Carlo Casari --- \textit{Politecnico di Milano, Milano, Italia}}
%\cvline{}{Dr. XYZ --- \textit{Universidad Polit\'ecnica de Madrid, Madrid, Espa\~na}}
%
\cvline{Description}{\small Atomic Layer Deposition (ALD) has been shown to be capable of depositing ultra-thin films (<100 nm) of perovskite oxides. These are technologically valuable due to their wide range of properties, ranging from ferroelectricity and ferromagnetism to superconductivity. With the use of ALD, it becomes possible to deposit thin and conformal coatings of these materials to three-dimensional surfaces with the ability to carefully control thickness with sub-nanometer resolution. Research work focuses on the lead titanate (PbTiO$_3$) end group of the PbTi$_{\mathrm{x}}$Zr$_{1-\mathrm{x}}$O$_{3}$ system.}

%%%%%%%%%% EXPERIENCE %%%%%%%%%%%%%%%%%

\section{Experience}

\cventry{Summer 2012}{Thesis Research}{Universidad Polit\'ecnica de Madrid}{Madrid}{Spain}{Three month period to be spent completing experiments and authoring final M.S. thesis document and defense presentation.}

\cventry{Fall, 2011}{Abstract Accepted to MRS}{}{}{}{On results of oriented lead titanate thin film deposition via ALD onto single crystalline surfaces.} 
            
\cventry{Spring, Summer 2011}{Cooperative Work Experience}{MesoMaterials Laboratory}{Drexel University}{}{Three six month periods (Spring, Summer) of full-time work as a student researcher. Focused research on application of atomic layer deposition to ferroelectric oxide films. }               

\cventry{2011}{Student Research}{MesoMaterials Laboratory}{Drexel University}{}{XXXX designing and conducting experiments concurrently while attending classes. Primary goals: Utilizing statistics to improve experimental designs; Optimizing process conditions based on predictive modeling and thermodynamic simulation; Identification of alternative chemical precursors to alleviate operational limitations. }

\cventry{Spring, Summer 2010}{Cooperative Work Experience}{MesoMaterials Laboratory}{Drexel University}{}{Three six month periods (Spring, Summer) of full-time work as a student researcher. Focused research on application of atomic layer deposition to ferroelectric oxide films. }   

\cventry{2010}{Student Research}{MesoMaterials Laboratory}{Drexel University}{}{XXXX designing and conducting experiments concurrently while attending classes. Primary goals: Utilizing statistics to improve experimental designs; Optimizing process conditions based on predictive modeling and thermodynamic simulation; Identification of alternative chemical precursors to alleviate operational limitations. }

\cventry{Spring, Summer 2009}{Cooperative Work Experience}{MesoMaterials Laboratory}{Drexel University}{}{Three six month periods (Spring, Summer) of full-time work as a student researcher. Focused research on application of atomic layer deposition to ferroelectric oxide films. }   

\cventry{2008-2009}{Student Research}{MesoMaterials Laboratory}{Drexel University}{}{XXXX designing and conducting experiments concurrently while attending classes. Primary goals: Utilizing statistics to improve experimental designs; Optimizing process conditions based on predictive modeling and thermodynamic simulation; Identification of alternative chemical precursors to alleviate operational limitations. }

\cventry{Summer 2008}{STAR: Summer Research Experience}{MesoMaterials Laboratory}{Drexel University}{}{Ten week period spent learning to organize and develop a personal research project. Focused on statistics-based design of experiments; data analysis using Matlab and Igor software packages and custom scripts; received training for and logged over 200 hours on characterization equipment, including: ellipsometry, scanning electron microscopy and EDS spectroscopy, X-Ray diffractometry, and Rutherford Backscattering Spectroscopy.}    

\cventry{2007--2008}{Training in Atomic Layer Deposition}{MesoMaterials Laboratory}{Drexel University}{}{Worked under graduate mentor Rahul Joseph researching application of advanced thin film deposition methods. Trained on operation and maintenance of ALD reactor, }


%%%%%%%% PUBLICATIONS %%%%%%%%%%%%%%%
              
\section{Publications}
\cventry{April, 2012}{Shape-Controlled Vapor-Transport Growth of Tellurium Nanowires}{Christopher J. Hawley, Brian R. Beatty, Guannan Chen, and Jonathan E. Spanier}{J. Crystal Growth \& Design, \textbf{ASAP}}{}{}

%%%%%%%% EQUIPMENT TRAINING %%%%%%%%%%%%%%%%

\section{Equipment Training}
\section{Equipment Experience}
\cventry{\textbf{Thin Film Deposition}}{\textmd{Atomic Layer Deposition (ALD)}}{\textup{Chemical Vapor Deposition (CVD), Pulsed Laser Deposition (PLD), \mbox{Molecular} Beam Epitaxy (MBE)}}{}{}{}

\cventry{\textbf{Film Quantification}}{\textmd{X-Ray Diffraction (XRD)}}{\textup{X-Ray Reflectivity (XRR), Scanning Electron Microscopy (SEM), Energy-Dispersive X-Ray Spectroscopy (EDXS), X-Ray Fluorescence (XRF), Ellipsometry, Rutherford Backscattering \mbox{Spectroscopy}}}{}{}{}

\cventry{\textbf{Chem., Therm. Analysis}}{\textmd{Fourier-Transform Infrared Spectroscopy (FTIR)}}{\textup{Differential Scanning Calorimetry (DSC),   
	\mbox{Thermo-gravimetric} Analysis (TGA)}}{}{}{}

%%%%%%%%%%%%%%%%%%%%%%%%%%%%%%%%%%%%%

%\section{Equipment Training} 
%\cvcomputer{\textbf{Thin Film Deposition}}{Atomic Layer Deposition (ALD), Chemical Vapor Deposition (CVD), Pulsed Laser Deposition (PLD), Molecular Beam %Epitaxy (MBE)} {}{}
%\cvcomputer{\textbf{Film Quantification}}{, X-Ray Reflectivity (XRR), Scanning Electron Microscopy (SEM), Energy-Dispersive X-Ray Spectroscopy (EDXS), X-Ray %Fluorescence (XRF), Ellipsometry, Rutherford Backscattering Spectroscopy (RBS)}{}{} 
%\cvcomputer{\textbf{Chemical and Thermal Analysis}}{Fourier-Transform Infrared Spectroscopy (FTIR), Differential Scanning Calorimetry (DSC),   Thermo-%gravimetric Analysis (TGA)}{}{} 
%\cvcomputer{}{} {}{}
%\vspace{10 mm}

\section{Computer skills} 
\cvcomputer{Languages}{MATLAB, Maple, \LaTeX} {}{}
\cvcomputer{Tools}{Origin, Igor, FilmWizard, Microsoft Office Suite, Adobe Suite} {} {}
\cvcomputer{Platforms}{Mac OS, Windows}  {}{} 
%\cvcomputer{}{} {}{}
%\vspace{10 mm} 

\section{Spoken Languages}
\cvlanguage{English}{Native}{} 
\cvlanguage{English}{Native}{}  \vspace{0.5em}

\cvlanguage{Spanish}{Speaking: Moderate \\ Reading: Elementary  \\ Writing: Elementary}{} \vspace{5mm}
\cvlanguage{Spanish}{Speaking: Moderate \\ Reading: Elementary  \\ Writing: Elementary}{} \vspace{0.5em}

\cvlanguage{Italian}{Speaking: Elementary \\ Reading: Elementary  \\ Writing: Elementary}{}

\section{Interests}
\cvline{Electronics}{\small Design and synthesis of novel materials for IC applications, particularly those leveraging unique quantum effects} 
\cvline{Energy}{\small Nanoscale structures for high-efficiency photovoltaic devices. }
\cvline{Culinary Arts}{\small Personal hobby.}

\closesection{}                   % needed to renewcommands
\renewcommand{\listitemsymbol}{-} % change the symbol for lists


%\section{Extra 1}
%\cvlistitem{Item 1}
%\cvlistitem{Item 2}
%\cvlistitem[+]{Item 3}            % optional other symbol

%\section{Extra 2}
%\cvlistdoubleitem[\Neutral]{Item 1}{Item 4}
%\cvlistdoubleitem[\Neutral]{Item 2}{Item 5}
%cvlistdoubleitem[\Neutral]{Item 3}{}

% Publications from a BibTeX file
%\nocite{*}
%\bibliographystyle{plain}
%\bibliography{publications}       % 'publications' is the name of a BibTeX file

\end{document}


%-----       letter       ---------------------------------------------------------
\clearpage
% recipient data
\recipient{Company Recruitment team}{Company, Inc.\\123 somestreet\\some city}
\date{January 01, 1984}
\opening{Dear Sir or Madam,}
\closing{Sincerely,}
\enclosure{Curriculum Vit\ae{}}
\makelettertitle

Lorem ipsum dolor sit amet, consectetur adipiscing elit. Duis ullamcorper neque sit amet lectus facilisis sed luctus nisl iaculis. Vivamus at neque arcu, sed tempor quam. Curabitur pharetra tincidunt tincidunt. Morbi volutpat feugiat mauris, quis tempor neque vehicula volutpat. Duis tristique justo vel massa fermentum accumsan. Mauris ante elit, feugiat vestibulum tempor eget, eleifend ac ipsum. Donec scelerisque lobortis ipsum eu vestibulum. Pellentesque vel massa at felis accumsan rhoncus.

Suspendisse commodo, massa eu congue tincidunt, elit mauris pellentesque orci, cursus tempor odio nisl euismod augue. Aliquam adipiscing nibh ut odio sodales et pulvinar tortor laoreet. Mauris a accumsan ligula. Class aptent taciti sociosqu ad litora torquent per conubia nostra, per inceptos himenaeos. Suspendisse vulputate sem vehicula ipsum varius nec tempus dui dapibus. Phasellus et est urna, ut auctor erat. Sed tincidunt odio id odio aliquam mattis. Donec sapien nulla, feugiat eget adipiscing sit amet, lacinia ut dolor. Phasellus tincidunt, leo a fringilla consectetur, felis diam aliquam urna, vitae aliquet lectus orci nec velit. Vivamus dapibus varius blandit.

Duis sit amet magna ante, at sodales diam. Aenean consectetur porta risus et sagittis. Ut interdum, enim varius pellentesque tincidunt, magna libero sodales tortor, ut fermentum nunc metus a ante. Vivamus odio leo, tincidunt eu luctus ut, sollicitudin sit amet metus. Nunc sed orci lectus. Ut sodales magna sed velit volutpat sit amet pulvinar diam venenatis.

\makeletterclosing
\clearpage


\end{document}
%% end of file `template_en.tex'.
